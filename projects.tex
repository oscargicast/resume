\section{Proyectos}

\cventry{2014-2015+}{\href{https://www.ventorystacj.com/}{Ventorystack}}{}{}{}
{A través de mi empresa, DGNEST SAC, automatizamos el proceso de inventario utilizando tecnologías móviles RFID en las instituciones públicas como \href{http://www.inictel-uni.edu.pe/}{INICTEL} y \href{http://en.wikipedia.org/wiki/National_University_of_Engineering}{UNI}.}   

\cventry{2014-2015}{\href{http://en.cronicas-upch.pe/allillanchu-proyecto-de-integracion-de-la-salud-mental-en-servicios-de-atencion-primaria/}{Allillanchu}}{}{}{}
{Trabajando en conjunto con la \href{http://www.upch.edu.pe/portal/}{Universidad Peruana Cayetano Heredia}. Desarrollamos una aplicación web y una plataforma móvil dedicada al estudio de la salud mental usando el cuestionario PHQ-9.}  

\cventry{Noviembre 2014}{\href{https://github.com/dgnest/museando}{Museando}}{OpenHack}{Hackspace}{Lima}
{Aplicación móvil para la interacción con museos a través de etiquetas RFID y tecnologías NFC.}  

\cventry{Noviembre 2014}{\href{https://github.com/dgnest/hospital}{Hospital}}{}{}{Lima}
{proyecto de código abierto utilizado Django para administrar los medicamentos de hospitales.}  

\cventry{2013+}{\href{http://dgnest.com}{dgnest.com}}{}{}{}
{Implementación de plataforma web para compartir proyectos de hardware de código abierto.}  

\cventry{2013}{\href{https://play.google.com/store/apps/details?id=com.sda.directorio&hl=en}{LIMA GUIDE}}{}{}{}
{Aplicación que brinda una guía de los números de emergencia de Lima tales como serenazgo, policía, bomberos, entre otros.
Además podrás encontrar los números telefónicos, dirección, página web, e-mail, etc. de algunas entidades públicas del estado tales como los ministerios, defensorías, SUNAT, RENIEC, y más.}  

\cventry{2013}{\href{https://play.google.com/store/apps/details?id=com.sda.tastebit&hl=en}{tasteBit}}{}{}{}
{TasteBit es una deliciosa y entretenida aplicación que te guiará a través de una ruta con mucho sabor. 
TasteBit te ayuda a probar las mejores cosas de la vida en tan solo un toque.}  

\cventry{2013}{\href{http://robotstudio.wdfiles.com/local--files/academics/Android.pdf}{New Sense}}{}{}{}
{Diseño e emplementación de una interfaz que reconocen patrones de movimiento del brazo basados en el sensado de datos de un dispositivo móvil Android.}

\cventry{Agosto 2012}{\href{http://robotstudio.wdfiles.com/local--files/academics/RobotsAutonomos.pdf}{Diseño de un Robot Móvil Ackerman}}{}{}{}
{Estimación de posición de un robot móvil tipo Ackerman utilizando el filtro extendido de Kalman.} 

\cventry{Setiembre 2011}{\href{http://robot-lab.blogspot.com/2012/09/pisco-sour-vending-machine.html}{Máquina dispensadora de Pisco Sour}}{}{}{}
{Este proyecto se realizó con el fin de promover el consumo de pisco sour y estandarizar este producto con el propósito de difundir el pisco peruano en el extranjero.}